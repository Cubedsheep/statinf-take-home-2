\documentclass[a4paper]{article}
\setlength{\parskip}{11pt plus 1pt minus 1pt}

%% Language and font encodings
\usepackage[dutch]{babel}
\usepackage[utf8]{inputenc}
\usepackage[T1]{fontenc} 
\usepackage[backend=bibtex]{biblatex}
\usepackage{url}
% \addbibresource{bibliografie.bib}
\usepackage[usenames,dvipsnames]{color}


%% Sets page size and margins
\usepackage[a4paper,top=3cm,bottom=2cm,left=1in]{geometry}

%% Useful packages
\usepackage{amsmath}
\usepackage{wasysym}
\usepackage{amssymb}
\usepackage[utf8]{inputenc}
\usepackage[dutch]{babel}
\usepackage{graphicx}
\usepackage[colorinlistoftodos]{todonotes}
\usepackage[colorlinks=false, allcolors=blue]{hyperref}
\usepackage{booktabs}
\usepackage{float}
\usepackage{siunitx}
\usepackage[backend=bibtex]{biblatex}
\usepackage{subcaption}
\usepackage{gensymb}

%% tikz for plots
\usepackage{tikz}
\usetikzlibrary{external}
\tikzexternalize[prefix=plots/]
% package to determine line width
\usepackage{layouts}

\newcommand{\R}{\mathbb{R}}

\title{Take home 2}
\author{Pieter Luyten}

\begin{document}

\maketitle

\section*{Question 1}
\subsection*{(a)}
The value for the intercept of the fit is $1.0321764$ and for the rico of the fit is $0.1904323$
\begin{figure}[h]
	\centering
	% Created by tikzDevice version 0.12.3 on 2019-12-14 20:27:39
% !TEX encoding = UTF-8 Unicode
\begin{tikzpicture}[x=1pt,y=1pt]
\definecolor{fillColor}{RGB}{255,255,255}
\path[use as bounding box,fill=fillColor,fill opacity=0.00] (0,0) rectangle (392.13,304.99);
\begin{scope}
\path[clip] ( 49.20, 61.20) rectangle (366.93,255.79);
\definecolor{drawColor}{RGB}{0,0,0}
\definecolor{fillColor}{RGB}{0,0,0}

\path[draw=drawColor,line width= 0.4pt,line join=round,line cap=round,fill=fillColor] ( 60.97,160.22) circle (  1.50);

\path[draw=drawColor,line width= 0.4pt,line join=round,line cap=round,fill=fillColor] ( 67.18, 96.94) circle (  1.50);

\path[draw=drawColor,line width= 0.4pt,line join=round,line cap=round,fill=fillColor] ( 76.43,109.00) circle (  1.50);

\path[draw=drawColor,line width= 0.4pt,line join=round,line cap=round,fill=fillColor] ( 89.08,130.08) circle (  1.50);

\path[draw=drawColor,line width= 0.4pt,line join=round,line cap=round,fill=fillColor] ( 89.92,137.76) circle (  1.50);

\path[draw=drawColor,line width= 0.4pt,line join=round,line cap=round,fill=fillColor] (104.06,132.96) circle (  1.50);

\path[draw=drawColor,line width= 0.4pt,line join=round,line cap=round,fill=fillColor] (106.85,134.30) circle (  1.50);

\path[draw=drawColor,line width= 0.4pt,line join=round,line cap=round,fill=fillColor] (110.69,122.39) circle (  1.50);

\path[draw=drawColor,line width= 0.4pt,line join=round,line cap=round,fill=fillColor] (113.15,144.58) circle (  1.50);

\path[draw=drawColor,line width= 0.4pt,line join=round,line cap=round,fill=fillColor] (129.00,164.86) circle (  1.50);

\path[draw=drawColor,line width= 0.4pt,line join=round,line cap=round,fill=fillColor] (131.28,169.93) circle (  1.50);

\path[draw=drawColor,line width= 0.4pt,line join=round,line cap=round,fill=fillColor] (136.73,158.46) circle (  1.50);

\path[draw=drawColor,line width= 0.4pt,line join=round,line cap=round,fill=fillColor] (139.70,184.68) circle (  1.50);

\path[draw=drawColor,line width= 0.4pt,line join=round,line cap=round,fill=fillColor] (139.75,127.86) circle (  1.50);

\path[draw=drawColor,line width= 0.4pt,line join=round,line cap=round,fill=fillColor] (153.06, 78.70) circle (  1.50);

\path[draw=drawColor,line width= 0.4pt,line join=round,line cap=round,fill=fillColor] (155.65,183.02) circle (  1.50);

\path[draw=drawColor,line width= 0.4pt,line join=round,line cap=round,fill=fillColor] (155.77,137.33) circle (  1.50);

\path[draw=drawColor,line width= 0.4pt,line join=round,line cap=round,fill=fillColor] (167.07,127.87) circle (  1.50);

\path[draw=drawColor,line width= 0.4pt,line join=round,line cap=round,fill=fillColor] (167.72,144.54) circle (  1.50);

\path[draw=drawColor,line width= 0.4pt,line join=round,line cap=round,fill=fillColor] (168.50,195.10) circle (  1.50);

\path[draw=drawColor,line width= 0.4pt,line join=round,line cap=round,fill=fillColor] (176.11,159.76) circle (  1.50);

\path[draw=drawColor,line width= 0.4pt,line join=round,line cap=round,fill=fillColor] (181.82,155.59) circle (  1.50);

\path[draw=drawColor,line width= 0.4pt,line join=round,line cap=round,fill=fillColor] (182.73,167.54) circle (  1.50);

\path[draw=drawColor,line width= 0.4pt,line join=round,line cap=round,fill=fillColor] (193.28,136.68) circle (  1.50);

\path[draw=drawColor,line width= 0.4pt,line join=round,line cap=round,fill=fillColor] (195.23,146.54) circle (  1.50);

\path[draw=drawColor,line width= 0.4pt,line join=round,line cap=round,fill=fillColor] (203.35,139.76) circle (  1.50);

\path[draw=drawColor,line width= 0.4pt,line join=round,line cap=round,fill=fillColor] (206.39,168.82) circle (  1.50);

\path[draw=drawColor,line width= 0.4pt,line join=round,line cap=round,fill=fillColor] (212.11,146.33) circle (  1.50);

\path[draw=drawColor,line width= 0.4pt,line join=round,line cap=round,fill=fillColor] (217.63,159.23) circle (  1.50);

\path[draw=drawColor,line width= 0.4pt,line join=round,line cap=round,fill=fillColor] (227.16,144.62) circle (  1.50);

\path[draw=drawColor,line width= 0.4pt,line join=round,line cap=round,fill=fillColor] (227.62,211.65) circle (  1.50);

\path[draw=drawColor,line width= 0.4pt,line join=round,line cap=round,fill=fillColor] (230.53,164.24) circle (  1.50);

\path[draw=drawColor,line width= 0.4pt,line join=round,line cap=round,fill=fillColor] (230.71,184.31) circle (  1.50);

\path[draw=drawColor,line width= 0.4pt,line join=round,line cap=round,fill=fillColor] (238.15,135.16) circle (  1.50);

\path[draw=drawColor,line width= 0.4pt,line join=round,line cap=round,fill=fillColor] (239.00,214.61) circle (  1.50);

\path[draw=drawColor,line width= 0.4pt,line join=round,line cap=round,fill=fillColor] (257.67,154.47) circle (  1.50);

\path[draw=drawColor,line width= 0.4pt,line join=round,line cap=round,fill=fillColor] (258.95,161.21) circle (  1.50);

\path[draw=drawColor,line width= 0.4pt,line join=round,line cap=round,fill=fillColor] (263.57,142.91) circle (  1.50);

\path[draw=drawColor,line width= 0.4pt,line join=round,line cap=round,fill=fillColor] (264.26,171.19) circle (  1.50);

\path[draw=drawColor,line width= 0.4pt,line join=round,line cap=round,fill=fillColor] (265.83,172.38) circle (  1.50);

\path[draw=drawColor,line width= 0.4pt,line join=round,line cap=round,fill=fillColor] (278.68,155.57) circle (  1.50);

\path[draw=drawColor,line width= 0.4pt,line join=round,line cap=round,fill=fillColor] (279.18,156.72) circle (  1.50);

\path[draw=drawColor,line width= 0.4pt,line join=round,line cap=round,fill=fillColor] (282.81,196.42) circle (  1.50);

\path[draw=drawColor,line width= 0.4pt,line join=round,line cap=round,fill=fillColor] (282.87,137.26) circle (  1.50);

\path[draw=drawColor,line width= 0.4pt,line join=round,line cap=round,fill=fillColor] (286.59,158.47) circle (  1.50);

\path[draw=drawColor,line width= 0.4pt,line join=round,line cap=round,fill=fillColor] (289.88,172.60) circle (  1.50);

\path[draw=drawColor,line width= 0.4pt,line join=round,line cap=round,fill=fillColor] (293.32,155.89) circle (  1.50);

\path[draw=drawColor,line width= 0.4pt,line join=round,line cap=round,fill=fillColor] (300.66,159.59) circle (  1.50);

\path[draw=drawColor,line width= 0.4pt,line join=round,line cap=round,fill=fillColor] (302.12,145.32) circle (  1.50);

\path[draw=drawColor,line width= 0.4pt,line join=round,line cap=round,fill=fillColor] (302.44,134.20) circle (  1.50);

\path[draw=drawColor,line width= 0.4pt,line join=round,line cap=round,fill=fillColor] (305.09, 92.69) circle (  1.50);

\path[draw=drawColor,line width= 0.4pt,line join=round,line cap=round,fill=fillColor] (310.65,196.93) circle (  1.50);

\path[draw=drawColor,line width= 0.4pt,line join=round,line cap=round,fill=fillColor] (319.30,174.25) circle (  1.50);

\path[draw=drawColor,line width= 0.4pt,line join=round,line cap=round,fill=fillColor] (326.01,136.52) circle (  1.50);

\path[draw=drawColor,line width= 0.4pt,line join=round,line cap=round,fill=fillColor] (330.59,141.99) circle (  1.50);

\path[draw=drawColor,line width= 0.4pt,line join=round,line cap=round,fill=fillColor] (336.81,159.42) circle (  1.50);

\path[draw=drawColor,line width= 0.4pt,line join=round,line cap=round,fill=fillColor] (345.67,151.60) circle (  1.50);

\path[draw=drawColor,line width= 0.4pt,line join=round,line cap=round,fill=fillColor] (348.53,208.52) circle (  1.50);

\path[draw=drawColor,line width= 0.4pt,line join=round,line cap=round,fill=fillColor] (349.98,158.22) circle (  1.50);

\path[draw=drawColor,line width= 0.4pt,line join=round,line cap=round,fill=fillColor] (355.16,179.72) circle (  1.50);
\end{scope}
\begin{scope}
\path[clip] (  0.00,  0.00) rectangle (392.13,304.99);
\definecolor{drawColor}{RGB}{0,0,0}

\path[draw=drawColor,line width= 0.4pt,line join=round,line cap=round] ( 60.50, 61.20) -- (355.17, 61.20);

\path[draw=drawColor,line width= 0.4pt,line join=round,line cap=round] ( 60.50, 61.20) -- ( 60.50, 55.20);

\path[draw=drawColor,line width= 0.4pt,line join=round,line cap=round] (134.17, 61.20) -- (134.17, 55.20);

\path[draw=drawColor,line width= 0.4pt,line join=round,line cap=round] (207.83, 61.20) -- (207.83, 55.20);

\path[draw=drawColor,line width= 0.4pt,line join=round,line cap=round] (281.50, 61.20) -- (281.50, 55.20);

\path[draw=drawColor,line width= 0.4pt,line join=round,line cap=round] (355.17, 61.20) -- (355.17, 55.20);

\node[text=drawColor,anchor=base,inner sep=0pt, outer sep=0pt, scale=  1.00] at ( 60.50, 39.60) {-1.0};

\node[text=drawColor,anchor=base,inner sep=0pt, outer sep=0pt, scale=  1.00] at (134.17, 39.60) {-0.5};

\node[text=drawColor,anchor=base,inner sep=0pt, outer sep=0pt, scale=  1.00] at (207.83, 39.60) {0.0};

\node[text=drawColor,anchor=base,inner sep=0pt, outer sep=0pt, scale=  1.00] at (281.50, 39.60) {0.5};

\node[text=drawColor,anchor=base,inner sep=0pt, outer sep=0pt, scale=  1.00] at (355.17, 39.60) {1.0};

\path[draw=drawColor,line width= 0.4pt,line join=round,line cap=round] ( 49.20, 68.41) -- ( 49.20,232.20);

\path[draw=drawColor,line width= 0.4pt,line join=round,line cap=round] ( 49.20, 68.41) -- ( 43.20, 68.41);

\path[draw=drawColor,line width= 0.4pt,line join=round,line cap=round] ( 49.20,109.36) -- ( 43.20,109.36);

\path[draw=drawColor,line width= 0.4pt,line join=round,line cap=round] ( 49.20,150.30) -- ( 43.20,150.30);

\path[draw=drawColor,line width= 0.4pt,line join=round,line cap=round] ( 49.20,191.25) -- ( 43.20,191.25);

\path[draw=drawColor,line width= 0.4pt,line join=round,line cap=round] ( 49.20,232.20) -- ( 43.20,232.20);

\node[text=drawColor,rotate= 90.00,anchor=base,inner sep=0pt, outer sep=0pt, scale=  1.00] at ( 34.80, 68.41) {0.0};

\node[text=drawColor,rotate= 90.00,anchor=base,inner sep=0pt, outer sep=0pt, scale=  1.00] at ( 34.80,109.36) {0.5};

\node[text=drawColor,rotate= 90.00,anchor=base,inner sep=0pt, outer sep=0pt, scale=  1.00] at ( 34.80,150.30) {1.0};

\node[text=drawColor,rotate= 90.00,anchor=base,inner sep=0pt, outer sep=0pt, scale=  1.00] at ( 34.80,191.25) {1.5};

\node[text=drawColor,rotate= 90.00,anchor=base,inner sep=0pt, outer sep=0pt, scale=  1.00] at ( 34.80,232.20) {2.0};

\path[draw=drawColor,line width= 0.4pt,line join=round,line cap=round] ( 49.20, 61.20) --
	(366.93, 61.20) --
	(366.93,255.79) --
	( 49.20,255.79) --
	( 49.20, 61.20);
\end{scope}
\begin{scope}
\path[clip] (  0.00,  0.00) rectangle (392.13,304.99);
\definecolor{drawColor}{RGB}{0,0,0}

\node[text=drawColor,anchor=base,inner sep=0pt, outer sep=0pt, scale=  1.00] at (208.06, 15.60) {V1};

\node[text=drawColor,rotate= 90.00,anchor=base,inner sep=0pt, outer sep=0pt, scale=  1.00] at ( 10.80,158.49) {V2};
\end{scope}
\begin{scope}
\path[clip] ( 49.20, 61.20) rectangle (366.93,255.79);
\definecolor{drawColor}{RGB}{0,0,0}

\path[draw=drawColor,line width= 0.4pt,line join=round,line cap=round] (  0.00,130.94) --
	( 10.54,132.05) --
	( 33.78,134.51) --
	( 57.02,136.97) --
	( 80.25,139.43) --
	(103.49,141.89) --
	(126.73,144.35) --
	(149.97,146.81) --
	(173.21,149.27) --
	(196.44,151.73) --
	(219.68,154.19) --
	(242.92,156.65) --
	(266.16,159.11) --
	(289.40,161.57) --
	(312.63,164.03) --
	(335.87,166.49) --
	(359.11,168.95) --
	(382.35,171.41) --
	(392.13,172.45);
\end{scope}
\end{tikzpicture}

	\caption{line fit throught the data in Ex1.txt}
	\label{fig:fit-1a}
\end{figure}

\subsection*{(b)}
Using the result from section 7.4.1 in <REFERENCE cursus> we know that the random variable
\begin{equation}
	T = \frac{\beta_1}{\sqrt{ \frac{S^2}{ \sum_{i=1}^{n}(x_i-\bar{x}^2)}}}
\end{equation}
has a Student distribution with $n-2$ degrees of freedom. The test value is $2.603$. Using a student-t distribution with $60-2=58$ degrees of freedom we find a p-value of $0.0117$. At the confidence level $\alpha=0.01$, the null hypothesis that $\beta_1=0$ holds. The $99\%$ confidence region for the test value is $[-2.663, 2.663]$.

\subsection*{(c)}
A q-q plot is a plot where the quantiles of the assumed distribution are plotted against the quantiles from the sample. So in a sample of $n$ points where we label the observation $x_i, i \in \{1,2, \ldots, n\}$ from lowest to highest, the ith point will be plotted at the $(Q(i/n), x_i)$. 
Here $Q(x)$ is the quantile function of the assumed distribution. This is a function such that $ P( X < Q(p)) = p$. If the assumed distribution is a good model for the observed sample, the points will well fitted by a linear function.

\begin{figure}[h]
	\centering
	% Created by tikzDevice version 0.12.3 on 2019-12-10 21:17:06
% !TEX encoding = UTF-8 Unicode
\begin{tikzpicture}[x=1pt,y=1pt]
\definecolor{fillColor}{RGB}{255,255,255}
\path[use as bounding box,fill=fillColor,fill opacity=0.00] (0,0) rectangle (392.13,304.99);
\begin{scope}
\path[clip] ( 49.20, 61.20) rectangle (366.93,255.79);
\definecolor{drawColor}{RGB}{0,0,0}

\path[draw=drawColor,line width= 0.4pt,line join=round,line cap=round] (265.49,198.91) circle (  2.25);

\path[draw=drawColor,line width= 0.4pt,line join=round,line cap=round] (101.66,109.54) circle (  2.25);

\path[draw=drawColor,line width= 0.4pt,line join=round,line cap=round] (111.66,125.02) circle (  2.25);

\path[draw=drawColor,line width= 0.4pt,line join=round,line cap=round] (177.31,152.62) circle (  2.25);

\path[draw=drawColor,line width= 0.4pt,line join=round,line cap=round] (217.08,163.23) circle (  2.25);

\path[draw=drawColor,line width= 0.4pt,line join=round,line cap=round] (185.76,154.42) circle (  2.25);

\path[draw=drawColor,line width= 0.4pt,line join=round,line cap=round] (188.48,155.88) circle (  2.25);

\path[draw=drawColor,line width= 0.4pt,line join=round,line cap=round] (150.64,138.68) circle (  2.25);

\path[draw=drawColor,line width= 0.4pt,line join=round,line cap=round] (219.68,169.33) circle (  2.25);

\path[draw=drawColor,line width= 0.4pt,line join=round,line cap=round] (261.63,195.32) circle (  2.25);

\path[draw=drawColor,line width= 0.4pt,line join=round,line cap=round] (269.59,202.07) circle (  2.25);

\path[draw=drawColor,line width= 0.4pt,line join=round,line cap=round] (247.91,185.23) circle (  2.25);

\path[draw=drawColor,line width= 0.4pt,line join=round,line cap=round] (296.51,221.44) circle (  2.25);

\path[draw=drawColor,line width= 0.4pt,line join=round,line cap=round] (154.50,142.02) circle (  2.25);

\path[draw=drawColor,line width= 0.4pt,line join=round,line cap=round] ( 87.64, 71.34) circle (  2.25);

\path[draw=drawColor,line width= 0.4pt,line join=round,line cap=round] (289.82,216.76) circle (  2.25);

\path[draw=drawColor,line width= 0.4pt,line join=round,line cap=round] (180.18,152.88) circle (  2.25);

\path[draw=drawColor,line width= 0.4pt,line join=round,line cap=round] (146.54,137.99) circle (  2.25);

\path[draw=drawColor,line width= 0.4pt,line join=round,line cap=round] (204.21,161.19) circle (  2.25);

\path[draw=drawColor,line width= 0.4pt,line join=round,line cap=round] (314.46,231.74) circle (  2.25);

\path[draw=drawColor,line width= 0.4pt,line join=round,line cap=round] (235.94,181.23) circle (  2.25);

\path[draw=drawColor,line width= 0.4pt,line join=round,line cap=round] (227.64,174.56) circle (  2.25);

\path[draw=drawColor,line width= 0.4pt,line join=round,line cap=round] (257.97,191.12) circle (  2.25);

\path[draw=drawColor,line width= 0.4pt,line join=round,line cap=round] (168.22,146.43) circle (  2.25);

\path[draw=drawColor,line width= 0.4pt,line join=round,line cap=round] (199.05,159.93) circle (  2.25);

\path[draw=drawColor,line width= 0.4pt,line join=round,line cap=round] (171.33,149.25) circle (  2.25);

\path[draw=drawColor,line width= 0.4pt,line join=round,line cap=round] (254.48,189.41) circle (  2.25);

\path[draw=drawColor,line width= 0.4pt,line join=round,line cap=round] (193.82,157.12) circle (  2.25);

\path[draw=drawColor,line width= 0.4pt,line join=round,line cap=round] (224.96,174.34) circle (  2.25);

\path[draw=drawColor,line width= 0.4pt,line join=round,line cap=round] (174.36,152.51) circle (  2.25);

\path[draw=drawColor,line width= 0.4pt,line join=round,line cap=round] (328.49,246.13) circle (  2.25);

\path[draw=drawColor,line width= 0.4pt,line join=round,line cap=round] (230.36,179.43) circle (  2.25);

\path[draw=drawColor,line width= 0.4pt,line join=round,line cap=round] (273.98,207.46) circle (  2.25);

\path[draw=drawColor,line width= 0.4pt,line join=round,line cap=round] (142.14,137.66) circle (  2.25);

\path[draw=drawColor,line width= 0.4pt,line join=round,line cap=round] (355.16,248.58) circle (  2.25);

\path[draw=drawColor,line width= 0.4pt,line join=round,line cap=round] (209.35,161.77) circle (  2.25);

\path[draw=drawColor,line width= 0.4pt,line join=round,line cap=round] (222.31,171.00) circle (  2.25);

\path[draw=drawColor,line width= 0.4pt,line join=round,line cap=round] (164.99,144.74) circle (  2.25);

\path[draw=drawColor,line width= 0.4pt,line join=round,line cap=round] (244.79,184.16) circle (  2.25);

\path[draw=drawColor,line width= 0.4pt,line join=round,line cap=round] (251.13,185.59) circle (  2.25);

\path[draw=drawColor,line width= 0.4pt,line join=round,line cap=round] (201.63,160.19) circle (  2.25);

\path[draw=drawColor,line width= 0.4pt,line join=round,line cap=round] (206.78,161.73) circle (  2.25);

\path[draw=drawColor,line width= 0.4pt,line join=round,line cap=round] (283.97,216.68) circle (  2.25);

\path[draw=drawColor,line width= 0.4pt,line join=round,line cap=round] (137.38,133.99) circle (  2.25);

\path[draw=drawColor,line width= 0.4pt,line join=round,line cap=round] (214.49,163.07) circle (  2.25);

\path[draw=drawColor,line width= 0.4pt,line join=round,line cap=round] (238.82,182.34) circle (  2.25);

\path[draw=drawColor,line width= 0.4pt,line join=round,line cap=round] (196.44,158.47) circle (  2.25);

\path[draw=drawColor,line width= 0.4pt,line join=round,line cap=round] (211.92,162.56) circle (  2.25);

\path[draw=drawColor,line width= 0.4pt,line join=round,line cap=round] (158.16,142.40) circle (  2.25);

\path[draw=drawColor,line width= 0.4pt,line join=round,line cap=round] (126.31,126.81) circle (  2.25);

\path[draw=drawColor,line width= 0.4pt,line join=round,line cap=round] ( 60.97, 68.41) circle (  2.25);

\path[draw=drawColor,line width= 0.4pt,line join=round,line cap=round] (278.74,213.27) circle (  2.25);

\path[draw=drawColor,line width= 0.4pt,line join=round,line cap=round] (233.13,180.30) circle (  2.25);

\path[draw=drawColor,line width= 0.4pt,line join=round,line cap=round] (119.61,126.57) circle (  2.25);

\path[draw=drawColor,line width= 0.4pt,line join=round,line cap=round] (132.15,133.53) circle (  2.25);

\path[draw=drawColor,line width= 0.4pt,line join=round,line cap=round] (191.17,156.97) circle (  2.25);

\path[draw=drawColor,line width= 0.4pt,line join=round,line cap=round] (161.65,144.73) circle (  2.25);

\path[draw=drawColor,line width= 0.4pt,line join=round,line cap=round] (304.46,223.86) circle (  2.25);

\path[draw=drawColor,line width= 0.4pt,line join=round,line cap=round] (183.00,153.35) circle (  2.25);

\path[draw=drawColor,line width= 0.4pt,line join=round,line cap=round] (241.77,182.64) circle (  2.25);
\end{scope}
\begin{scope}
\path[clip] (  0.00,  0.00) rectangle (392.13,304.99);
\definecolor{drawColor}{RGB}{0,0,0}

\path[draw=drawColor,line width= 0.4pt,line join=round,line cap=round] ( 85.18, 61.20) -- (330.95, 61.20);

\path[draw=drawColor,line width= 0.4pt,line join=round,line cap=round] ( 85.18, 61.20) -- ( 85.18, 55.20);

\path[draw=drawColor,line width= 0.4pt,line join=round,line cap=round] (146.62, 61.20) -- (146.62, 55.20);

\path[draw=drawColor,line width= 0.4pt,line join=round,line cap=round] (208.06, 61.20) -- (208.06, 55.20);

\path[draw=drawColor,line width= 0.4pt,line join=round,line cap=round] (269.51, 61.20) -- (269.51, 55.20);

\path[draw=drawColor,line width= 0.4pt,line join=round,line cap=round] (330.95, 61.20) -- (330.95, 55.20);

\node[text=drawColor,anchor=base,inner sep=0pt, outer sep=0pt, scale=  1.00] at ( 85.18, 39.60) {-2};

\node[text=drawColor,anchor=base,inner sep=0pt, outer sep=0pt, scale=  1.00] at (146.62, 39.60) {-1};

\node[text=drawColor,anchor=base,inner sep=0pt, outer sep=0pt, scale=  1.00] at (208.06, 39.60) {0};

\node[text=drawColor,anchor=base,inner sep=0pt, outer sep=0pt, scale=  1.00] at (269.51, 39.60) {1};

\node[text=drawColor,anchor=base,inner sep=0pt, outer sep=0pt, scale=  1.00] at (330.95, 39.60) {2};

\path[draw=drawColor,line width= 0.4pt,line join=round,line cap=round] ( 49.20,109.77) -- ( 49.20,224.23);

\path[draw=drawColor,line width= 0.4pt,line join=round,line cap=round] ( 49.20,109.77) -- ( 43.20,109.77);

\path[draw=drawColor,line width= 0.4pt,line join=round,line cap=round] ( 49.20,167.00) -- ( 43.20,167.00);

\path[draw=drawColor,line width= 0.4pt,line join=round,line cap=round] ( 49.20,224.23) -- ( 43.20,224.23);

\node[text=drawColor,rotate= 90.00,anchor=base,inner sep=0pt, outer sep=0pt, scale=  1.00] at ( 34.80,109.77) {-0.5};

\node[text=drawColor,rotate= 90.00,anchor=base,inner sep=0pt, outer sep=0pt, scale=  1.00] at ( 34.80,167.00) {0.0};

\node[text=drawColor,rotate= 90.00,anchor=base,inner sep=0pt, outer sep=0pt, scale=  1.00] at ( 34.80,224.23) {0.5};

\path[draw=drawColor,line width= 0.4pt,line join=round,line cap=round] ( 49.20, 61.20) --
	(366.93, 61.20) --
	(366.93,255.79) --
	( 49.20,255.79) --
	( 49.20, 61.20);
\end{scope}
\begin{scope}
\path[clip] (  0.00,  0.00) rectangle (392.13,304.99);
\definecolor{drawColor}{RGB}{0,0,0}

\node[text=drawColor,anchor=base,inner sep=0pt, outer sep=0pt, scale=  1.20] at (208.06,276.25) {\bfseries Normal Q-Q Plot};

\node[text=drawColor,anchor=base,inner sep=0pt, outer sep=0pt, scale=  1.00] at (208.06, 15.60) {Theoretical Quantiles};

\node[text=drawColor,rotate= 90.00,anchor=base,inner sep=0pt, outer sep=0pt, scale=  1.00] at ( 10.80,158.49) {Sample Quantiles};
\end{scope}
\begin{scope}
\path[clip] ( 49.20, 61.20) rectangle (366.93,255.79);
\definecolor{drawColor}{RGB}{0,0,0}

\path[draw=drawColor,line width= 0.4pt,line join=round,line cap=round] ( 49.20, 90.31) -- (366.93,241.01);
\end{scope}
\end{tikzpicture}

	\caption{qq plot for the errors on the fit}
	\label{fig:qqplot-1c}
\end{figure}

%\printinunitsof{in}\prntlen{\textwidth}


\end{document}
