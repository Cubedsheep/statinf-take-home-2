\documentclass[a4paper]{article}
\setlength{\parskip}{11pt plus 1pt minus 1pt}

%% Language and font encodings
\usepackage[dutch]{babel}
\usepackage[utf8]{inputenc}
\usepackage[T1]{fontenc} 
\usepackage[backend=bibtex]{biblatex}
\usepackage{url}
% \addbibresource{bibliografie.bib}
\usepackage[usenames,dvipsnames]{color}


%% Sets page size and margins
\usepackage[a4paper,top=3cm,bottom=2cm,left=1in]{geometry}

%% Useful packages
\usepackage{amsmath}
\usepackage{wasysym}
\usepackage{amssymb}
\usepackage[utf8]{inputenc}
\usepackage[dutch]{babel}
\usepackage{graphicx}
\usepackage[colorinlistoftodos]{todonotes}
\usepackage[colorlinks=false, allcolors=blue]{hyperref}
\usepackage{booktabs}
\usepackage{float}
\usepackage{siunitx}
\usepackage[backend=bibtex]{biblatex}
\usepackage{subcaption}
\usepackage{gensymb}

%% tikz for plots
\usepackage{tikz}
\usetikzlibrary{external}
\tikzexternalize[prefix=plots/]
% package to determine line width
\usepackage{layouts}

\newcommand{\R}{\mathbb{R}}

\title{Take home 2}
\author{Pieter Luyten}

\begin{document}

\maketitle

\section*{Question 1}
\subsection*{(a)}
The value for the intercept of the fit is $1.0321764$ and for the rico of the fit is $0.1904323$
\begin{figure}[h]
	\centering
	% Created by tikzDevice version 0.12.3 on 2019-12-14 20:27:39
% !TEX encoding = UTF-8 Unicode
\begin{tikzpicture}[x=1pt,y=1pt]
\definecolor{fillColor}{RGB}{255,255,255}
\path[use as bounding box,fill=fillColor,fill opacity=0.00] (0,0) rectangle (392.13,304.99);
\begin{scope}
\path[clip] ( 49.20, 61.20) rectangle (366.93,255.79);
\definecolor{drawColor}{RGB}{0,0,0}
\definecolor{fillColor}{RGB}{0,0,0}

\path[draw=drawColor,line width= 0.4pt,line join=round,line cap=round,fill=fillColor] ( 60.97,160.22) circle (  1.50);

\path[draw=drawColor,line width= 0.4pt,line join=round,line cap=round,fill=fillColor] ( 67.18, 96.94) circle (  1.50);

\path[draw=drawColor,line width= 0.4pt,line join=round,line cap=round,fill=fillColor] ( 76.43,109.00) circle (  1.50);

\path[draw=drawColor,line width= 0.4pt,line join=round,line cap=round,fill=fillColor] ( 89.08,130.08) circle (  1.50);

\path[draw=drawColor,line width= 0.4pt,line join=round,line cap=round,fill=fillColor] ( 89.92,137.76) circle (  1.50);

\path[draw=drawColor,line width= 0.4pt,line join=round,line cap=round,fill=fillColor] (104.06,132.96) circle (  1.50);

\path[draw=drawColor,line width= 0.4pt,line join=round,line cap=round,fill=fillColor] (106.85,134.30) circle (  1.50);

\path[draw=drawColor,line width= 0.4pt,line join=round,line cap=round,fill=fillColor] (110.69,122.39) circle (  1.50);

\path[draw=drawColor,line width= 0.4pt,line join=round,line cap=round,fill=fillColor] (113.15,144.58) circle (  1.50);

\path[draw=drawColor,line width= 0.4pt,line join=round,line cap=round,fill=fillColor] (129.00,164.86) circle (  1.50);

\path[draw=drawColor,line width= 0.4pt,line join=round,line cap=round,fill=fillColor] (131.28,169.93) circle (  1.50);

\path[draw=drawColor,line width= 0.4pt,line join=round,line cap=round,fill=fillColor] (136.73,158.46) circle (  1.50);

\path[draw=drawColor,line width= 0.4pt,line join=round,line cap=round,fill=fillColor] (139.70,184.68) circle (  1.50);

\path[draw=drawColor,line width= 0.4pt,line join=round,line cap=round,fill=fillColor] (139.75,127.86) circle (  1.50);

\path[draw=drawColor,line width= 0.4pt,line join=round,line cap=round,fill=fillColor] (153.06, 78.70) circle (  1.50);

\path[draw=drawColor,line width= 0.4pt,line join=round,line cap=round,fill=fillColor] (155.65,183.02) circle (  1.50);

\path[draw=drawColor,line width= 0.4pt,line join=round,line cap=round,fill=fillColor] (155.77,137.33) circle (  1.50);

\path[draw=drawColor,line width= 0.4pt,line join=round,line cap=round,fill=fillColor] (167.07,127.87) circle (  1.50);

\path[draw=drawColor,line width= 0.4pt,line join=round,line cap=round,fill=fillColor] (167.72,144.54) circle (  1.50);

\path[draw=drawColor,line width= 0.4pt,line join=round,line cap=round,fill=fillColor] (168.50,195.10) circle (  1.50);

\path[draw=drawColor,line width= 0.4pt,line join=round,line cap=round,fill=fillColor] (176.11,159.76) circle (  1.50);

\path[draw=drawColor,line width= 0.4pt,line join=round,line cap=round,fill=fillColor] (181.82,155.59) circle (  1.50);

\path[draw=drawColor,line width= 0.4pt,line join=round,line cap=round,fill=fillColor] (182.73,167.54) circle (  1.50);

\path[draw=drawColor,line width= 0.4pt,line join=round,line cap=round,fill=fillColor] (193.28,136.68) circle (  1.50);

\path[draw=drawColor,line width= 0.4pt,line join=round,line cap=round,fill=fillColor] (195.23,146.54) circle (  1.50);

\path[draw=drawColor,line width= 0.4pt,line join=round,line cap=round,fill=fillColor] (203.35,139.76) circle (  1.50);

\path[draw=drawColor,line width= 0.4pt,line join=round,line cap=round,fill=fillColor] (206.39,168.82) circle (  1.50);

\path[draw=drawColor,line width= 0.4pt,line join=round,line cap=round,fill=fillColor] (212.11,146.33) circle (  1.50);

\path[draw=drawColor,line width= 0.4pt,line join=round,line cap=round,fill=fillColor] (217.63,159.23) circle (  1.50);

\path[draw=drawColor,line width= 0.4pt,line join=round,line cap=round,fill=fillColor] (227.16,144.62) circle (  1.50);

\path[draw=drawColor,line width= 0.4pt,line join=round,line cap=round,fill=fillColor] (227.62,211.65) circle (  1.50);

\path[draw=drawColor,line width= 0.4pt,line join=round,line cap=round,fill=fillColor] (230.53,164.24) circle (  1.50);

\path[draw=drawColor,line width= 0.4pt,line join=round,line cap=round,fill=fillColor] (230.71,184.31) circle (  1.50);

\path[draw=drawColor,line width= 0.4pt,line join=round,line cap=round,fill=fillColor] (238.15,135.16) circle (  1.50);

\path[draw=drawColor,line width= 0.4pt,line join=round,line cap=round,fill=fillColor] (239.00,214.61) circle (  1.50);

\path[draw=drawColor,line width= 0.4pt,line join=round,line cap=round,fill=fillColor] (257.67,154.47) circle (  1.50);

\path[draw=drawColor,line width= 0.4pt,line join=round,line cap=round,fill=fillColor] (258.95,161.21) circle (  1.50);

\path[draw=drawColor,line width= 0.4pt,line join=round,line cap=round,fill=fillColor] (263.57,142.91) circle (  1.50);

\path[draw=drawColor,line width= 0.4pt,line join=round,line cap=round,fill=fillColor] (264.26,171.19) circle (  1.50);

\path[draw=drawColor,line width= 0.4pt,line join=round,line cap=round,fill=fillColor] (265.83,172.38) circle (  1.50);

\path[draw=drawColor,line width= 0.4pt,line join=round,line cap=round,fill=fillColor] (278.68,155.57) circle (  1.50);

\path[draw=drawColor,line width= 0.4pt,line join=round,line cap=round,fill=fillColor] (279.18,156.72) circle (  1.50);

\path[draw=drawColor,line width= 0.4pt,line join=round,line cap=round,fill=fillColor] (282.81,196.42) circle (  1.50);

\path[draw=drawColor,line width= 0.4pt,line join=round,line cap=round,fill=fillColor] (282.87,137.26) circle (  1.50);

\path[draw=drawColor,line width= 0.4pt,line join=round,line cap=round,fill=fillColor] (286.59,158.47) circle (  1.50);

\path[draw=drawColor,line width= 0.4pt,line join=round,line cap=round,fill=fillColor] (289.88,172.60) circle (  1.50);

\path[draw=drawColor,line width= 0.4pt,line join=round,line cap=round,fill=fillColor] (293.32,155.89) circle (  1.50);

\path[draw=drawColor,line width= 0.4pt,line join=round,line cap=round,fill=fillColor] (300.66,159.59) circle (  1.50);

\path[draw=drawColor,line width= 0.4pt,line join=round,line cap=round,fill=fillColor] (302.12,145.32) circle (  1.50);

\path[draw=drawColor,line width= 0.4pt,line join=round,line cap=round,fill=fillColor] (302.44,134.20) circle (  1.50);

\path[draw=drawColor,line width= 0.4pt,line join=round,line cap=round,fill=fillColor] (305.09, 92.69) circle (  1.50);

\path[draw=drawColor,line width= 0.4pt,line join=round,line cap=round,fill=fillColor] (310.65,196.93) circle (  1.50);

\path[draw=drawColor,line width= 0.4pt,line join=round,line cap=round,fill=fillColor] (319.30,174.25) circle (  1.50);

\path[draw=drawColor,line width= 0.4pt,line join=round,line cap=round,fill=fillColor] (326.01,136.52) circle (  1.50);

\path[draw=drawColor,line width= 0.4pt,line join=round,line cap=round,fill=fillColor] (330.59,141.99) circle (  1.50);

\path[draw=drawColor,line width= 0.4pt,line join=round,line cap=round,fill=fillColor] (336.81,159.42) circle (  1.50);

\path[draw=drawColor,line width= 0.4pt,line join=round,line cap=round,fill=fillColor] (345.67,151.60) circle (  1.50);

\path[draw=drawColor,line width= 0.4pt,line join=round,line cap=round,fill=fillColor] (348.53,208.52) circle (  1.50);

\path[draw=drawColor,line width= 0.4pt,line join=round,line cap=round,fill=fillColor] (349.98,158.22) circle (  1.50);

\path[draw=drawColor,line width= 0.4pt,line join=round,line cap=round,fill=fillColor] (355.16,179.72) circle (  1.50);
\end{scope}
\begin{scope}
\path[clip] (  0.00,  0.00) rectangle (392.13,304.99);
\definecolor{drawColor}{RGB}{0,0,0}

\path[draw=drawColor,line width= 0.4pt,line join=round,line cap=round] ( 60.50, 61.20) -- (355.17, 61.20);

\path[draw=drawColor,line width= 0.4pt,line join=round,line cap=round] ( 60.50, 61.20) -- ( 60.50, 55.20);

\path[draw=drawColor,line width= 0.4pt,line join=round,line cap=round] (134.17, 61.20) -- (134.17, 55.20);

\path[draw=drawColor,line width= 0.4pt,line join=round,line cap=round] (207.83, 61.20) -- (207.83, 55.20);

\path[draw=drawColor,line width= 0.4pt,line join=round,line cap=round] (281.50, 61.20) -- (281.50, 55.20);

\path[draw=drawColor,line width= 0.4pt,line join=round,line cap=round] (355.17, 61.20) -- (355.17, 55.20);

\node[text=drawColor,anchor=base,inner sep=0pt, outer sep=0pt, scale=  1.00] at ( 60.50, 39.60) {-1.0};

\node[text=drawColor,anchor=base,inner sep=0pt, outer sep=0pt, scale=  1.00] at (134.17, 39.60) {-0.5};

\node[text=drawColor,anchor=base,inner sep=0pt, outer sep=0pt, scale=  1.00] at (207.83, 39.60) {0.0};

\node[text=drawColor,anchor=base,inner sep=0pt, outer sep=0pt, scale=  1.00] at (281.50, 39.60) {0.5};

\node[text=drawColor,anchor=base,inner sep=0pt, outer sep=0pt, scale=  1.00] at (355.17, 39.60) {1.0};

\path[draw=drawColor,line width= 0.4pt,line join=round,line cap=round] ( 49.20, 68.41) -- ( 49.20,232.20);

\path[draw=drawColor,line width= 0.4pt,line join=round,line cap=round] ( 49.20, 68.41) -- ( 43.20, 68.41);

\path[draw=drawColor,line width= 0.4pt,line join=round,line cap=round] ( 49.20,109.36) -- ( 43.20,109.36);

\path[draw=drawColor,line width= 0.4pt,line join=round,line cap=round] ( 49.20,150.30) -- ( 43.20,150.30);

\path[draw=drawColor,line width= 0.4pt,line join=round,line cap=round] ( 49.20,191.25) -- ( 43.20,191.25);

\path[draw=drawColor,line width= 0.4pt,line join=round,line cap=round] ( 49.20,232.20) -- ( 43.20,232.20);

\node[text=drawColor,rotate= 90.00,anchor=base,inner sep=0pt, outer sep=0pt, scale=  1.00] at ( 34.80, 68.41) {0.0};

\node[text=drawColor,rotate= 90.00,anchor=base,inner sep=0pt, outer sep=0pt, scale=  1.00] at ( 34.80,109.36) {0.5};

\node[text=drawColor,rotate= 90.00,anchor=base,inner sep=0pt, outer sep=0pt, scale=  1.00] at ( 34.80,150.30) {1.0};

\node[text=drawColor,rotate= 90.00,anchor=base,inner sep=0pt, outer sep=0pt, scale=  1.00] at ( 34.80,191.25) {1.5};

\node[text=drawColor,rotate= 90.00,anchor=base,inner sep=0pt, outer sep=0pt, scale=  1.00] at ( 34.80,232.20) {2.0};

\path[draw=drawColor,line width= 0.4pt,line join=round,line cap=round] ( 49.20, 61.20) --
	(366.93, 61.20) --
	(366.93,255.79) --
	( 49.20,255.79) --
	( 49.20, 61.20);
\end{scope}
\begin{scope}
\path[clip] (  0.00,  0.00) rectangle (392.13,304.99);
\definecolor{drawColor}{RGB}{0,0,0}

\node[text=drawColor,anchor=base,inner sep=0pt, outer sep=0pt, scale=  1.00] at (208.06, 15.60) {V1};

\node[text=drawColor,rotate= 90.00,anchor=base,inner sep=0pt, outer sep=0pt, scale=  1.00] at ( 10.80,158.49) {V2};
\end{scope}
\begin{scope}
\path[clip] ( 49.20, 61.20) rectangle (366.93,255.79);
\definecolor{drawColor}{RGB}{0,0,0}

\path[draw=drawColor,line width= 0.4pt,line join=round,line cap=round] (  0.00,130.94) --
	( 10.54,132.05) --
	( 33.78,134.51) --
	( 57.02,136.97) --
	( 80.25,139.43) --
	(103.49,141.89) --
	(126.73,144.35) --
	(149.97,146.81) --
	(173.21,149.27) --
	(196.44,151.73) --
	(219.68,154.19) --
	(242.92,156.65) --
	(266.16,159.11) --
	(289.40,161.57) --
	(312.63,164.03) --
	(335.87,166.49) --
	(359.11,168.95) --
	(382.35,171.41) --
	(392.13,172.45);
\end{scope}
\end{tikzpicture}

	\caption{line fit throught the data in Ex1.txt}
	\label{fig:fit-1a}
\end{figure}

\subsection*{(b)}
Using the result from section 7.4.1 in <REFERENCE cursus> we know that the random variable
\begin{equation}
	T = \frac{\beta_1}{\sqrt{ \frac{S^2}{ \sum_{i=1}^{n}(x_i-\bar{x}^2)}}}
\end{equation}
has a Student distribution with $n-2$ degrees of freedom. The test value is $2.603$. Using a student-t distribution with $60-2=58$ degrees of freedom we find a p-value of $0.0117$. At the confidence level $\alpha=0.01$, the null hypothesis that $\beta_1=0$ holds. The $99\%$ confidence region for the test value is $[-2.663, 2.663]$.

\subsection*{(c)}
A q-q plot is a plot where the quantiles of the assumed distribution are plotted against the quantiles from the sample. So in a sample of $n$ points where we label the observation $x_i, i \in \{1,2, \ldots, n\}$ from lowest to highest, the ith point will be plotted at the $(Q(i/n), x_i)$. 
Here $Q(x)$ is the quantile function of the assumed distribution. This is a function such that $ P( X < Q(p)) = p$. If the assumed distribution is a good model for the observed sample, the points will well fitted by a linear function.

\begin{figure}[h]
	\centering
	% Created by tikzDevice version 0.12.3 on 2019-12-10 21:17:06
% !TEX encoding = UTF-8 Unicode
\begin{tikzpicture}[x=1pt,y=1pt]
\definecolor{fillColor}{RGB}{255,255,255}
\path[use as bounding box,fill=fillColor,fill opacity=0.00] (0,0) rectangle (392.13,304.99);
\begin{scope}
\path[clip] ( 49.20, 61.20) rectangle (366.93,255.79);
\definecolor{drawColor}{RGB}{0,0,0}

\path[draw=drawColor,line width= 0.4pt,line join=round,line cap=round] (265.49,198.91) circle (  2.25);

\path[draw=drawColor,line width= 0.4pt,line join=round,line cap=round] (101.66,109.54) circle (  2.25);

\path[draw=drawColor,line width= 0.4pt,line join=round,line cap=round] (111.66,125.02) circle (  2.25);

\path[draw=drawColor,line width= 0.4pt,line join=round,line cap=round] (177.31,152.62) circle (  2.25);

\path[draw=drawColor,line width= 0.4pt,line join=round,line cap=round] (217.08,163.23) circle (  2.25);

\path[draw=drawColor,line width= 0.4pt,line join=round,line cap=round] (185.76,154.42) circle (  2.25);

\path[draw=drawColor,line width= 0.4pt,line join=round,line cap=round] (188.48,155.88) circle (  2.25);

\path[draw=drawColor,line width= 0.4pt,line join=round,line cap=round] (150.64,138.68) circle (  2.25);

\path[draw=drawColor,line width= 0.4pt,line join=round,line cap=round] (219.68,169.33) circle (  2.25);

\path[draw=drawColor,line width= 0.4pt,line join=round,line cap=round] (261.63,195.32) circle (  2.25);

\path[draw=drawColor,line width= 0.4pt,line join=round,line cap=round] (269.59,202.07) circle (  2.25);

\path[draw=drawColor,line width= 0.4pt,line join=round,line cap=round] (247.91,185.23) circle (  2.25);

\path[draw=drawColor,line width= 0.4pt,line join=round,line cap=round] (296.51,221.44) circle (  2.25);

\path[draw=drawColor,line width= 0.4pt,line join=round,line cap=round] (154.50,142.02) circle (  2.25);

\path[draw=drawColor,line width= 0.4pt,line join=round,line cap=round] ( 87.64, 71.34) circle (  2.25);

\path[draw=drawColor,line width= 0.4pt,line join=round,line cap=round] (289.82,216.76) circle (  2.25);

\path[draw=drawColor,line width= 0.4pt,line join=round,line cap=round] (180.18,152.88) circle (  2.25);

\path[draw=drawColor,line width= 0.4pt,line join=round,line cap=round] (146.54,137.99) circle (  2.25);

\path[draw=drawColor,line width= 0.4pt,line join=round,line cap=round] (204.21,161.19) circle (  2.25);

\path[draw=drawColor,line width= 0.4pt,line join=round,line cap=round] (314.46,231.74) circle (  2.25);

\path[draw=drawColor,line width= 0.4pt,line join=round,line cap=round] (235.94,181.23) circle (  2.25);

\path[draw=drawColor,line width= 0.4pt,line join=round,line cap=round] (227.64,174.56) circle (  2.25);

\path[draw=drawColor,line width= 0.4pt,line join=round,line cap=round] (257.97,191.12) circle (  2.25);

\path[draw=drawColor,line width= 0.4pt,line join=round,line cap=round] (168.22,146.43) circle (  2.25);

\path[draw=drawColor,line width= 0.4pt,line join=round,line cap=round] (199.05,159.93) circle (  2.25);

\path[draw=drawColor,line width= 0.4pt,line join=round,line cap=round] (171.33,149.25) circle (  2.25);

\path[draw=drawColor,line width= 0.4pt,line join=round,line cap=round] (254.48,189.41) circle (  2.25);

\path[draw=drawColor,line width= 0.4pt,line join=round,line cap=round] (193.82,157.12) circle (  2.25);

\path[draw=drawColor,line width= 0.4pt,line join=round,line cap=round] (224.96,174.34) circle (  2.25);

\path[draw=drawColor,line width= 0.4pt,line join=round,line cap=round] (174.36,152.51) circle (  2.25);

\path[draw=drawColor,line width= 0.4pt,line join=round,line cap=round] (328.49,246.13) circle (  2.25);

\path[draw=drawColor,line width= 0.4pt,line join=round,line cap=round] (230.36,179.43) circle (  2.25);

\path[draw=drawColor,line width= 0.4pt,line join=round,line cap=round] (273.98,207.46) circle (  2.25);

\path[draw=drawColor,line width= 0.4pt,line join=round,line cap=round] (142.14,137.66) circle (  2.25);

\path[draw=drawColor,line width= 0.4pt,line join=round,line cap=round] (355.16,248.58) circle (  2.25);

\path[draw=drawColor,line width= 0.4pt,line join=round,line cap=round] (209.35,161.77) circle (  2.25);

\path[draw=drawColor,line width= 0.4pt,line join=round,line cap=round] (222.31,171.00) circle (  2.25);

\path[draw=drawColor,line width= 0.4pt,line join=round,line cap=round] (164.99,144.74) circle (  2.25);

\path[draw=drawColor,line width= 0.4pt,line join=round,line cap=round] (244.79,184.16) circle (  2.25);

\path[draw=drawColor,line width= 0.4pt,line join=round,line cap=round] (251.13,185.59) circle (  2.25);

\path[draw=drawColor,line width= 0.4pt,line join=round,line cap=round] (201.63,160.19) circle (  2.25);

\path[draw=drawColor,line width= 0.4pt,line join=round,line cap=round] (206.78,161.73) circle (  2.25);

\path[draw=drawColor,line width= 0.4pt,line join=round,line cap=round] (283.97,216.68) circle (  2.25);

\path[draw=drawColor,line width= 0.4pt,line join=round,line cap=round] (137.38,133.99) circle (  2.25);

\path[draw=drawColor,line width= 0.4pt,line join=round,line cap=round] (214.49,163.07) circle (  2.25);

\path[draw=drawColor,line width= 0.4pt,line join=round,line cap=round] (238.82,182.34) circle (  2.25);

\path[draw=drawColor,line width= 0.4pt,line join=round,line cap=round] (196.44,158.47) circle (  2.25);

\path[draw=drawColor,line width= 0.4pt,line join=round,line cap=round] (211.92,162.56) circle (  2.25);

\path[draw=drawColor,line width= 0.4pt,line join=round,line cap=round] (158.16,142.40) circle (  2.25);

\path[draw=drawColor,line width= 0.4pt,line join=round,line cap=round] (126.31,126.81) circle (  2.25);

\path[draw=drawColor,line width= 0.4pt,line join=round,line cap=round] ( 60.97, 68.41) circle (  2.25);

\path[draw=drawColor,line width= 0.4pt,line join=round,line cap=round] (278.74,213.27) circle (  2.25);

\path[draw=drawColor,line width= 0.4pt,line join=round,line cap=round] (233.13,180.30) circle (  2.25);

\path[draw=drawColor,line width= 0.4pt,line join=round,line cap=round] (119.61,126.57) circle (  2.25);

\path[draw=drawColor,line width= 0.4pt,line join=round,line cap=round] (132.15,133.53) circle (  2.25);

\path[draw=drawColor,line width= 0.4pt,line join=round,line cap=round] (191.17,156.97) circle (  2.25);

\path[draw=drawColor,line width= 0.4pt,line join=round,line cap=round] (161.65,144.73) circle (  2.25);

\path[draw=drawColor,line width= 0.4pt,line join=round,line cap=round] (304.46,223.86) circle (  2.25);

\path[draw=drawColor,line width= 0.4pt,line join=round,line cap=round] (183.00,153.35) circle (  2.25);

\path[draw=drawColor,line width= 0.4pt,line join=round,line cap=round] (241.77,182.64) circle (  2.25);
\end{scope}
\begin{scope}
\path[clip] (  0.00,  0.00) rectangle (392.13,304.99);
\definecolor{drawColor}{RGB}{0,0,0}

\path[draw=drawColor,line width= 0.4pt,line join=round,line cap=round] ( 85.18, 61.20) -- (330.95, 61.20);

\path[draw=drawColor,line width= 0.4pt,line join=round,line cap=round] ( 85.18, 61.20) -- ( 85.18, 55.20);

\path[draw=drawColor,line width= 0.4pt,line join=round,line cap=round] (146.62, 61.20) -- (146.62, 55.20);

\path[draw=drawColor,line width= 0.4pt,line join=round,line cap=round] (208.06, 61.20) -- (208.06, 55.20);

\path[draw=drawColor,line width= 0.4pt,line join=round,line cap=round] (269.51, 61.20) -- (269.51, 55.20);

\path[draw=drawColor,line width= 0.4pt,line join=round,line cap=round] (330.95, 61.20) -- (330.95, 55.20);

\node[text=drawColor,anchor=base,inner sep=0pt, outer sep=0pt, scale=  1.00] at ( 85.18, 39.60) {-2};

\node[text=drawColor,anchor=base,inner sep=0pt, outer sep=0pt, scale=  1.00] at (146.62, 39.60) {-1};

\node[text=drawColor,anchor=base,inner sep=0pt, outer sep=0pt, scale=  1.00] at (208.06, 39.60) {0};

\node[text=drawColor,anchor=base,inner sep=0pt, outer sep=0pt, scale=  1.00] at (269.51, 39.60) {1};

\node[text=drawColor,anchor=base,inner sep=0pt, outer sep=0pt, scale=  1.00] at (330.95, 39.60) {2};

\path[draw=drawColor,line width= 0.4pt,line join=round,line cap=round] ( 49.20,109.77) -- ( 49.20,224.23);

\path[draw=drawColor,line width= 0.4pt,line join=round,line cap=round] ( 49.20,109.77) -- ( 43.20,109.77);

\path[draw=drawColor,line width= 0.4pt,line join=round,line cap=round] ( 49.20,167.00) -- ( 43.20,167.00);

\path[draw=drawColor,line width= 0.4pt,line join=round,line cap=round] ( 49.20,224.23) -- ( 43.20,224.23);

\node[text=drawColor,rotate= 90.00,anchor=base,inner sep=0pt, outer sep=0pt, scale=  1.00] at ( 34.80,109.77) {-0.5};

\node[text=drawColor,rotate= 90.00,anchor=base,inner sep=0pt, outer sep=0pt, scale=  1.00] at ( 34.80,167.00) {0.0};

\node[text=drawColor,rotate= 90.00,anchor=base,inner sep=0pt, outer sep=0pt, scale=  1.00] at ( 34.80,224.23) {0.5};

\path[draw=drawColor,line width= 0.4pt,line join=round,line cap=round] ( 49.20, 61.20) --
	(366.93, 61.20) --
	(366.93,255.79) --
	( 49.20,255.79) --
	( 49.20, 61.20);
\end{scope}
\begin{scope}
\path[clip] (  0.00,  0.00) rectangle (392.13,304.99);
\definecolor{drawColor}{RGB}{0,0,0}

\node[text=drawColor,anchor=base,inner sep=0pt, outer sep=0pt, scale=  1.20] at (208.06,276.25) {\bfseries Normal Q-Q Plot};

\node[text=drawColor,anchor=base,inner sep=0pt, outer sep=0pt, scale=  1.00] at (208.06, 15.60) {Theoretical Quantiles};

\node[text=drawColor,rotate= 90.00,anchor=base,inner sep=0pt, outer sep=0pt, scale=  1.00] at ( 10.80,158.49) {Sample Quantiles};
\end{scope}
\begin{scope}
\path[clip] ( 49.20, 61.20) rectangle (366.93,255.79);
\definecolor{drawColor}{RGB}{0,0,0}

\path[draw=drawColor,line width= 0.4pt,line join=round,line cap=round] ( 49.20, 90.31) -- (366.93,241.01);
\end{scope}
\end{tikzpicture}

	\caption{qq plot for the errors on the fit}
	\label{fig:qqplot-1c}
\end{figure}

\section*{Question 3}
\subsection*{(a)}

The values for the fitted coefficients with ordinary least squares are: $[0.4994, 1.1985, -2.4959, 1.2082]$. In figure \ref{fig:fit-3} the fit is plotted along with the data. 
\begin{figure}
	\centering
	% Created by tikzDevice version 0.12.3 on 2019-12-11 20:36:26
% !TEX encoding = UTF-8 Unicode
\begin{tikzpicture}[x=1pt,y=1pt]
\definecolor{fillColor}{RGB}{255,255,255}
\path[use as bounding box,fill=fillColor,fill opacity=0.00] (0,0) rectangle (392.13,304.99);
\begin{scope}
\path[clip] ( 49.20, 61.20) rectangle (366.93,255.79);
\definecolor{drawColor}{RGB}{0,255,0}
\definecolor{fillColor}{RGB}{0,255,0}

\path[draw=drawColor,line width= 0.4pt,line join=round,line cap=round,fill=fillColor] ( 60.97,168.42) circle (  1.50);

\path[draw=drawColor,line width= 0.4pt,line join=round,line cap=round,fill=fillColor] ( 63.44,169.00) circle (  1.50);

\path[draw=drawColor,line width= 0.4pt,line join=round,line cap=round,fill=fillColor] ( 65.91,169.36) circle (  1.50);

\path[draw=drawColor,line width= 0.4pt,line join=round,line cap=round,fill=fillColor] ( 68.38,169.83) circle (  1.50);

\path[draw=drawColor,line width= 0.4pt,line join=round,line cap=round,fill=fillColor] ( 70.86,170.57) circle (  1.50);

\path[draw=drawColor,line width= 0.4pt,line join=round,line cap=round,fill=fillColor] ( 73.33,171.04) circle (  1.50);

\path[draw=drawColor,line width= 0.4pt,line join=round,line cap=round,fill=fillColor] ( 75.80,171.11) circle (  1.50);

\path[draw=drawColor,line width= 0.4pt,line join=round,line cap=round,fill=fillColor] ( 78.27,171.51) circle (  1.50);

\path[draw=drawColor,line width= 0.4pt,line join=round,line cap=round,fill=fillColor] ( 80.75,170.98) circle (  1.50);

\path[draw=drawColor,line width= 0.4pt,line join=round,line cap=round,fill=fillColor] ( 83.22,171.11) circle (  1.50);

\path[draw=drawColor,line width= 0.4pt,line join=round,line cap=round,fill=fillColor] ( 85.69,170.84) circle (  1.50);

\path[draw=drawColor,line width= 0.4pt,line join=round,line cap=round,fill=fillColor] ( 88.16,169.89) circle (  1.50);

\path[draw=drawColor,line width= 0.4pt,line join=round,line cap=round,fill=fillColor] ( 90.63,173.21) circle (  1.50);

\path[draw=drawColor,line width= 0.4pt,line join=round,line cap=round,fill=fillColor] ( 93.11,174.38) circle (  1.50);

\path[draw=drawColor,line width= 0.4pt,line join=round,line cap=round,fill=fillColor] ( 95.58,174.45) circle (  1.50);

\path[draw=drawColor,line width= 0.4pt,line join=round,line cap=round,fill=fillColor] ( 98.05,171.87) circle (  1.50);

\path[draw=drawColor,line width= 0.4pt,line join=round,line cap=round,fill=fillColor] (100.52,170.74) circle (  1.50);

\path[draw=drawColor,line width= 0.4pt,line join=round,line cap=round,fill=fillColor] (102.99,172.39) circle (  1.50);

\path[draw=drawColor,line width= 0.4pt,line join=round,line cap=round,fill=fillColor] (105.47,180.23) circle (  1.50);

\path[draw=drawColor,line width= 0.4pt,line join=round,line cap=round,fill=fillColor] (107.94,168.91) circle (  1.50);

\path[draw=drawColor,line width= 0.4pt,line join=round,line cap=round,fill=fillColor] (110.41,173.95) circle (  1.50);

\path[draw=drawColor,line width= 0.4pt,line join=round,line cap=round,fill=fillColor] (112.88,183.76) circle (  1.50);

\path[draw=drawColor,line width= 0.4pt,line join=round,line cap=round,fill=fillColor] (115.36,169.56) circle (  1.50);

\path[draw=drawColor,line width= 0.4pt,line join=round,line cap=round,fill=fillColor] (117.83,177.93) circle (  1.50);

\path[draw=drawColor,line width= 0.4pt,line join=round,line cap=round,fill=fillColor] (120.30,174.85) circle (  1.50);

\path[draw=drawColor,line width= 0.4pt,line join=round,line cap=round,fill=fillColor] (122.77,173.32) circle (  1.50);

\path[draw=drawColor,line width= 0.4pt,line join=round,line cap=round,fill=fillColor] (125.24,163.92) circle (  1.50);

\path[draw=drawColor,line width= 0.4pt,line join=round,line cap=round,fill=fillColor] (127.72,176.44) circle (  1.50);

\path[draw=drawColor,line width= 0.4pt,line join=round,line cap=round,fill=fillColor] (130.19,160.03) circle (  1.50);

\path[draw=drawColor,line width= 0.4pt,line join=round,line cap=round,fill=fillColor] (132.66,185.25) circle (  1.50);

\path[draw=drawColor,line width= 0.4pt,line join=round,line cap=round,fill=fillColor] (135.13,174.50) circle (  1.50);

\path[draw=drawColor,line width= 0.4pt,line join=round,line cap=round,fill=fillColor] (137.61,194.59) circle (  1.50);

\path[draw=drawColor,line width= 0.4pt,line join=round,line cap=round,fill=fillColor] (140.08,169.23) circle (  1.50);

\path[draw=drawColor,line width= 0.4pt,line join=round,line cap=round,fill=fillColor] (142.55,167.61) circle (  1.50);

\path[draw=drawColor,line width= 0.4pt,line join=round,line cap=round,fill=fillColor] (145.02,180.61) circle (  1.50);

\path[draw=drawColor,line width= 0.4pt,line join=round,line cap=round,fill=fillColor] (147.49,170.89) circle (  1.50);

\path[draw=drawColor,line width= 0.4pt,line join=round,line cap=round,fill=fillColor] (149.97,164.41) circle (  1.50);

\path[draw=drawColor,line width= 0.4pt,line join=round,line cap=round,fill=fillColor] (152.44,178.31) circle (  1.50);

\path[draw=drawColor,line width= 0.4pt,line join=round,line cap=round,fill=fillColor] (154.91,190.57) circle (  1.50);

\path[draw=drawColor,line width= 0.4pt,line join=round,line cap=round,fill=fillColor] (157.38,167.10) circle (  1.50);

\path[draw=drawColor,line width= 0.4pt,line join=round,line cap=round,fill=fillColor] (159.86,170.14) circle (  1.50);

\path[draw=drawColor,line width= 0.4pt,line join=round,line cap=round,fill=fillColor] (162.33,188.47) circle (  1.50);

\path[draw=drawColor,line width= 0.4pt,line join=round,line cap=round,fill=fillColor] (164.80,155.18) circle (  1.50);

\path[draw=drawColor,line width= 0.4pt,line join=round,line cap=round,fill=fillColor] (167.27,197.79) circle (  1.50);

\path[draw=drawColor,line width= 0.4pt,line join=round,line cap=round,fill=fillColor] (169.74,186.79) circle (  1.50);

\path[draw=drawColor,line width= 0.4pt,line join=round,line cap=round,fill=fillColor] (172.22,164.72) circle (  1.50);

\path[draw=drawColor,line width= 0.4pt,line join=round,line cap=round,fill=fillColor] (174.69,106.11) circle (  1.50);

\path[draw=drawColor,line width= 0.4pt,line join=round,line cap=round,fill=fillColor] (177.16,168.06) circle (  1.50);

\path[draw=drawColor,line width= 0.4pt,line join=round,line cap=round,fill=fillColor] (179.63,152.30) circle (  1.50);

\path[draw=drawColor,line width= 0.4pt,line join=round,line cap=round,fill=fillColor] (182.11,160.71) circle (  1.50);

\path[draw=drawColor,line width= 0.4pt,line join=round,line cap=round,fill=fillColor] (184.58,164.54) circle (  1.50);

\path[draw=drawColor,line width= 0.4pt,line join=round,line cap=round,fill=fillColor] (187.05,154.65) circle (  1.50);

\path[draw=drawColor,line width= 0.4pt,line join=round,line cap=round,fill=fillColor] (189.52,185.03) circle (  1.50);

\path[draw=drawColor,line width= 0.4pt,line join=round,line cap=round,fill=fillColor] (191.99,190.32) circle (  1.50);

\path[draw=drawColor,line width= 0.4pt,line join=round,line cap=round,fill=fillColor] (194.47,203.25) circle (  1.50);

\path[draw=drawColor,line width= 0.4pt,line join=round,line cap=round,fill=fillColor] (196.94,196.82) circle (  1.50);

\path[draw=drawColor,line width= 0.4pt,line join=round,line cap=round,fill=fillColor] (199.41,160.09) circle (  1.50);

\path[draw=drawColor,line width= 0.4pt,line join=round,line cap=round,fill=fillColor] (201.88, 95.93) circle (  1.50);

\path[draw=drawColor,line width= 0.4pt,line join=round,line cap=round,fill=fillColor] (204.35,144.77) circle (  1.50);

\path[draw=drawColor,line width= 0.4pt,line join=round,line cap=round,fill=fillColor] (206.83,140.72) circle (  1.50);

\path[draw=drawColor,line width= 0.4pt,line join=round,line cap=round,fill=fillColor] (209.30,164.42) circle (  1.50);

\path[draw=drawColor,line width= 0.4pt,line join=round,line cap=round,fill=fillColor] (211.77,141.39) circle (  1.50);

\path[draw=drawColor,line width= 0.4pt,line join=round,line cap=round,fill=fillColor] (214.24,228.38) circle (  1.50);

\path[draw=drawColor,line width= 0.4pt,line join=round,line cap=round,fill=fillColor] (216.72,134.95) circle (  1.50);

\path[draw=drawColor,line width= 0.4pt,line join=round,line cap=round,fill=fillColor] (219.19,161.14) circle (  1.50);

\path[draw=drawColor,line width= 0.4pt,line join=round,line cap=round,fill=fillColor] (221.66,118.35) circle (  1.50);

\path[draw=drawColor,line width= 0.4pt,line join=round,line cap=round,fill=fillColor] (224.13,234.40) circle (  1.50);

\path[draw=drawColor,line width= 0.4pt,line join=round,line cap=round,fill=fillColor] (226.60,245.10) circle (  1.50);

\path[draw=drawColor,line width= 0.4pt,line join=round,line cap=round,fill=fillColor] (229.08,164.23) circle (  1.50);

\path[draw=drawColor,line width= 0.4pt,line join=round,line cap=round,fill=fillColor] (231.55,146.08) circle (  1.50);

\path[draw=drawColor,line width= 0.4pt,line join=round,line cap=round,fill=fillColor] (234.02,105.78) circle (  1.50);

\path[draw=drawColor,line width= 0.4pt,line join=round,line cap=round,fill=fillColor] (236.49,177.28) circle (  1.50);

\path[draw=drawColor,line width= 0.4pt,line join=round,line cap=round,fill=fillColor] (238.97, 68.41) circle (  1.50);

\path[draw=drawColor,line width= 0.4pt,line join=round,line cap=round,fill=fillColor] (241.44,109.00) circle (  1.50);

\path[draw=drawColor,line width= 0.4pt,line join=round,line cap=round,fill=fillColor] (243.91,221.04) circle (  1.50);

\path[draw=drawColor,line width= 0.4pt,line join=round,line cap=round,fill=fillColor] (246.38,159.66) circle (  1.50);

\path[draw=drawColor,line width= 0.4pt,line join=round,line cap=round,fill=fillColor] (248.85,151.47) circle (  1.50);

\path[draw=drawColor,line width= 0.4pt,line join=round,line cap=round,fill=fillColor] (251.33,166.43) circle (  1.50);

\path[draw=drawColor,line width= 0.4pt,line join=round,line cap=round,fill=fillColor] (253.80,146.73) circle (  1.50);

\path[draw=drawColor,line width= 0.4pt,line join=round,line cap=round,fill=fillColor] (256.27,205.56) circle (  1.50);

\path[draw=drawColor,line width= 0.4pt,line join=round,line cap=round,fill=fillColor] (258.74,171.77) circle (  1.50);

\path[draw=drawColor,line width= 0.4pt,line join=round,line cap=round,fill=fillColor] (261.22,213.76) circle (  1.50);

\path[draw=drawColor,line width= 0.4pt,line join=round,line cap=round,fill=fillColor] (263.69,179.22) circle (  1.50);

\path[draw=drawColor,line width= 0.4pt,line join=round,line cap=round,fill=fillColor] (266.16,130.02) circle (  1.50);

\path[draw=drawColor,line width= 0.4pt,line join=round,line cap=round,fill=fillColor] (268.63,147.63) circle (  1.50);

\path[draw=drawColor,line width= 0.4pt,line join=round,line cap=round,fill=fillColor] (271.10,222.61) circle (  1.50);

\path[draw=drawColor,line width= 0.4pt,line join=round,line cap=round,fill=fillColor] (273.58,248.58) circle (  1.50);

\path[draw=drawColor,line width= 0.4pt,line join=round,line cap=round,fill=fillColor] (276.05,164.94) circle (  1.50);

\path[draw=drawColor,line width= 0.4pt,line join=round,line cap=round,fill=fillColor] (278.52,151.85) circle (  1.50);

\path[draw=drawColor,line width= 0.4pt,line join=round,line cap=round,fill=fillColor] (280.99,243.46) circle (  1.50);

\path[draw=drawColor,line width= 0.4pt,line join=round,line cap=round,fill=fillColor] (283.46,127.97) circle (  1.50);

\path[draw=drawColor,line width= 0.4pt,line join=round,line cap=round,fill=fillColor] (285.94,184.54) circle (  1.50);

\path[draw=drawColor,line width= 0.4pt,line join=round,line cap=round,fill=fillColor] (288.41,185.97) circle (  1.50);

\path[draw=drawColor,line width= 0.4pt,line join=round,line cap=round,fill=fillColor] (290.88,190.33) circle (  1.50);

\path[draw=drawColor,line width= 0.4pt,line join=round,line cap=round,fill=fillColor] (293.35,172.41) circle (  1.50);

\path[draw=drawColor,line width= 0.4pt,line join=round,line cap=round,fill=fillColor] (295.83,201.31) circle (  1.50);

\path[draw=drawColor,line width= 0.4pt,line join=round,line cap=round,fill=fillColor] (298.30,189.48) circle (  1.50);

\path[draw=drawColor,line width= 0.4pt,line join=round,line cap=round,fill=fillColor] (300.77,178.48) circle (  1.50);

\path[draw=drawColor,line width= 0.4pt,line join=round,line cap=round,fill=fillColor] (303.24,164.24) circle (  1.50);

\path[draw=drawColor,line width= 0.4pt,line join=round,line cap=round,fill=fillColor] (305.71,193.33) circle (  1.50);

\path[draw=drawColor,line width= 0.4pt,line join=round,line cap=round,fill=fillColor] (308.19,196.00) circle (  1.50);

\path[draw=drawColor,line width= 0.4pt,line join=round,line cap=round,fill=fillColor] (310.66,199.24) circle (  1.50);

\path[draw=drawColor,line width= 0.4pt,line join=round,line cap=round,fill=fillColor] (313.13,186.53) circle (  1.50);

\path[draw=drawColor,line width= 0.4pt,line join=round,line cap=round,fill=fillColor] (315.60,182.58) circle (  1.50);

\path[draw=drawColor,line width= 0.4pt,line join=round,line cap=round,fill=fillColor] (318.08,202.74) circle (  1.50);

\path[draw=drawColor,line width= 0.4pt,line join=round,line cap=round,fill=fillColor] (320.55,216.00) circle (  1.50);

\path[draw=drawColor,line width= 0.4pt,line join=round,line cap=round,fill=fillColor] (323.02,204.43) circle (  1.50);

\path[draw=drawColor,line width= 0.4pt,line join=round,line cap=round,fill=fillColor] (325.49,208.80) circle (  1.50);

\path[draw=drawColor,line width= 0.4pt,line join=round,line cap=round,fill=fillColor] (327.96,207.23) circle (  1.50);

\path[draw=drawColor,line width= 0.4pt,line join=round,line cap=round,fill=fillColor] (330.44,212.38) circle (  1.50);

\path[draw=drawColor,line width= 0.4pt,line join=round,line cap=round,fill=fillColor] (332.91,221.30) circle (  1.50);

\path[draw=drawColor,line width= 0.4pt,line join=round,line cap=round,fill=fillColor] (335.38,219.07) circle (  1.50);

\path[draw=drawColor,line width= 0.4pt,line join=round,line cap=round,fill=fillColor] (337.85,215.87) circle (  1.50);

\path[draw=drawColor,line width= 0.4pt,line join=round,line cap=round,fill=fillColor] (340.33,221.50) circle (  1.50);

\path[draw=drawColor,line width= 0.4pt,line join=round,line cap=round,fill=fillColor] (342.80,225.55) circle (  1.50);

\path[draw=drawColor,line width= 0.4pt,line join=round,line cap=round,fill=fillColor] (345.27,227.72) circle (  1.50);

\path[draw=drawColor,line width= 0.4pt,line join=round,line cap=round,fill=fillColor] (347.74,229.77) circle (  1.50);

\path[draw=drawColor,line width= 0.4pt,line join=round,line cap=round,fill=fillColor] (350.21,233.27) circle (  1.50);

\path[draw=drawColor,line width= 0.4pt,line join=round,line cap=round,fill=fillColor] (352.69,235.96) circle (  1.50);

\path[draw=drawColor,line width= 0.4pt,line join=round,line cap=round,fill=fillColor] (355.16,238.85) circle (  1.50);
\end{scope}
\begin{scope}
\path[clip] (  0.00,  0.00) rectangle (392.13,304.99);
\definecolor{drawColor}{RGB}{0,0,0}

\path[draw=drawColor,line width= 0.4pt,line join=round,line cap=round] ( 59.48, 61.20) -- (356.64, 61.20);

\path[draw=drawColor,line width= 0.4pt,line join=round,line cap=round] ( 59.48, 61.20) -- ( 59.48, 55.20);

\path[draw=drawColor,line width= 0.4pt,line join=round,line cap=round] (133.77, 61.20) -- (133.77, 55.20);

\path[draw=drawColor,line width= 0.4pt,line join=round,line cap=round] (208.06, 61.20) -- (208.06, 55.20);

\path[draw=drawColor,line width= 0.4pt,line join=round,line cap=round] (282.35, 61.20) -- (282.35, 55.20);

\path[draw=drawColor,line width= 0.4pt,line join=round,line cap=round] (356.64, 61.20) -- (356.64, 55.20);

\node[text=drawColor,anchor=base,inner sep=0pt, outer sep=0pt, scale=  1.00] at ( 59.48, 39.60) {0.0};

\node[text=drawColor,anchor=base,inner sep=0pt, outer sep=0pt, scale=  1.00] at (133.77, 39.60) {0.5};

\node[text=drawColor,anchor=base,inner sep=0pt, outer sep=0pt, scale=  1.00] at (208.06, 39.60) {1.0};

\node[text=drawColor,anchor=base,inner sep=0pt, outer sep=0pt, scale=  1.00] at (282.35, 39.60) {1.5};

\node[text=drawColor,anchor=base,inner sep=0pt, outer sep=0pt, scale=  1.00] at (356.64, 39.60) {2.0};

\path[draw=drawColor,line width= 0.4pt,line join=round,line cap=round] ( 49.20, 77.33) -- ( 49.20,222.51);

\path[draw=drawColor,line width= 0.4pt,line join=round,line cap=round] ( 49.20, 77.33) -- ( 43.20, 77.33);

\path[draw=drawColor,line width= 0.4pt,line join=round,line cap=round] ( 49.20,113.63) -- ( 43.20,113.63);

\path[draw=drawColor,line width= 0.4pt,line join=round,line cap=round] ( 49.20,149.92) -- ( 43.20,149.92);

\path[draw=drawColor,line width= 0.4pt,line join=round,line cap=round] ( 49.20,186.21) -- ( 43.20,186.21);

\path[draw=drawColor,line width= 0.4pt,line join=round,line cap=round] ( 49.20,222.51) -- ( 43.20,222.51);

\node[text=drawColor,rotate= 90.00,anchor=base,inner sep=0pt, outer sep=0pt, scale=  1.00] at ( 34.80, 77.33) {-2};

\node[text=drawColor,rotate= 90.00,anchor=base,inner sep=0pt, outer sep=0pt, scale=  1.00] at ( 34.80,113.63) {-1};

\node[text=drawColor,rotate= 90.00,anchor=base,inner sep=0pt, outer sep=0pt, scale=  1.00] at ( 34.80,149.92) {0};

\node[text=drawColor,rotate= 90.00,anchor=base,inner sep=0pt, outer sep=0pt, scale=  1.00] at ( 34.80,186.21) {1};

\node[text=drawColor,rotate= 90.00,anchor=base,inner sep=0pt, outer sep=0pt, scale=  1.00] at ( 34.80,222.51) {2};

\path[draw=drawColor,line width= 0.4pt,line join=round,line cap=round] ( 49.20, 61.20) --
	(366.93, 61.20) --
	(366.93,255.79) --
	( 49.20,255.79) --
	( 49.20, 61.20);
\end{scope}
\begin{scope}
\path[clip] (  0.00,  0.00) rectangle (392.13,304.99);
\definecolor{drawColor}{RGB}{0,0,0}

\node[text=drawColor,anchor=base,inner sep=0pt, outer sep=0pt, scale=  1.00] at (208.06, 15.60) {V1};

\node[text=drawColor,rotate= 90.00,anchor=base,inner sep=0pt, outer sep=0pt, scale=  1.00] at ( 10.80,158.49) {V2};
\end{scope}
\begin{scope}
\path[clip] ( 49.20, 61.20) rectangle (366.93,255.79);
\definecolor{drawColor}{RGB}{255,0,0}

\path[draw=drawColor,line width= 0.4pt,line join=round,line cap=round] ( 46.11,163.36) --
	( 49.38,164.65) --
	( 52.65,165.85) --
	( 55.92,166.95) --
	( 59.20,167.96) --
	( 62.47,168.88) --
	( 65.74,169.72) --
	( 69.01,170.47) --
	( 72.28,171.15) --
	( 75.56,171.75) --
	( 78.83,172.27) --
	( 82.10,172.72) --
	( 85.37,173.11) --
	( 88.64,173.42) --
	( 91.91,173.68) --
	( 95.19,173.87) --
	( 98.46,174.01) --
	(101.73,174.10) --
	(105.00,174.13) --
	(108.27,174.11) --
	(111.55,174.05) --
	(114.82,173.94) --
	(118.09,173.80) --
	(121.36,173.62) --
	(124.63,173.40) --
	(127.90,173.15) --
	(131.18,172.87) --
	(134.45,172.56) --
	(137.72,172.23) --
	(140.99,171.88) --
	(144.26,171.52) --
	(147.53,171.13) --
	(150.81,170.74) --
	(154.08,170.34) --
	(157.35,169.93) --
	(160.62,169.51) --
	(163.89,169.09) --
	(167.17,168.68) --
	(170.44,168.27) --
	(173.71,167.87) --
	(176.98,167.48) --
	(180.25,167.10) --
	(183.52,166.74) --
	(186.80,166.39) --
	(190.07,166.07) --
	(193.34,165.77) --
	(196.61,165.50) --
	(199.88,165.26) --
	(203.16,165.05) --
	(206.43,164.88) --
	(209.70,164.74) --
	(212.97,164.65) --
	(216.24,164.60) --
	(219.51,164.60) --
	(222.79,164.64) --
	(226.06,164.74) --
	(229.33,164.90) --
	(232.60,165.11) --
	(235.87,165.38) --
	(239.15,165.72) --
	(242.42,166.12) --
	(245.69,166.59) --
	(248.96,167.14) --
	(252.23,167.76) --
	(255.50,168.45) --
	(258.78,169.23) --
	(262.05,170.09) --
	(265.32,171.04) --
	(268.59,172.07) --
	(271.86,173.20) --
	(275.13,174.42) --
	(278.41,175.74) --
	(281.68,177.16) --
	(284.95,178.68) --
	(288.22,180.31) --
	(291.49,182.04) --
	(294.77,183.89) --
	(298.04,185.85) --
	(301.31,187.93) --
	(304.58,190.13) --
	(307.85,192.45) --
	(311.12,194.90) --
	(314.40,197.47) --
	(317.67,200.18) --
	(320.94,203.02) --
	(324.21,205.99) --
	(327.48,209.11) --
	(330.76,212.37) --
	(334.03,215.77) --
	(337.30,219.32) --
	(340.57,223.02) --
	(343.84,226.88) --
	(347.11,230.89) --
	(350.39,235.06) --
	(353.66,239.39) --
	(356.93,243.89) --
	(360.20,248.55) --
	(363.47,253.39) --
	(366.74,258.40) --
	(370.02,263.58);
\definecolor{drawColor}{RGB}{0,0,255}

\path[draw=drawColor,line width= 0.4pt,line join=round,line cap=round] ( 46.11,164.32) --
	( 49.38,165.35) --
	( 52.65,166.31) --
	( 55.92,167.20) --
	( 59.20,168.01) --
	( 62.47,168.76) --
	( 65.74,169.43) --
	( 69.01,170.05) --
	( 72.28,170.60) --
	( 75.56,171.09) --
	( 78.83,171.52) --
	( 82.10,171.89) --
	( 85.37,172.22) --
	( 88.64,172.49) --
	( 91.91,172.71) --
	( 95.19,172.89) --
	( 98.46,173.02) --
	(101.73,173.12) --
	(105.00,173.17) --
	(108.27,173.18) --
	(111.55,173.16) --
	(114.82,173.11) --
	(118.09,173.03) --
	(121.36,172.92) --
	(124.63,172.78) --
	(127.90,172.62) --
	(131.18,172.44) --
	(134.45,172.24) --
	(137.72,172.03) --
	(140.99,171.80) --
	(144.26,171.56) --
	(147.53,171.31) --
	(150.81,171.05) --
	(154.08,170.79) --
	(157.35,170.52) --
	(160.62,170.26) --
	(163.89,169.99) --
	(167.17,169.73) --
	(170.44,169.48) --
	(173.71,169.24) --
	(176.98,169.00) --
	(180.25,168.79) --
	(183.52,168.58) --
	(186.80,168.40) --
	(190.07,168.23) --
	(193.34,168.09) --
	(196.61,167.98) --
	(199.88,167.89) --
	(203.16,167.83) --
	(206.43,167.80) --
	(209.70,167.81) --
	(212.97,167.85) --
	(216.24,167.94) --
	(219.51,168.06) --
	(222.79,168.23) --
	(226.06,168.44) --
	(229.33,168.70) --
	(232.60,169.01) --
	(235.87,169.38) --
	(239.15,169.80) --
	(242.42,170.27) --
	(245.69,170.81) --
	(248.96,171.41) --
	(252.23,172.07) --
	(255.50,172.80) --
	(258.78,173.60) --
	(262.05,174.47) --
	(265.32,175.41) --
	(268.59,176.43) --
	(271.86,177.53) --
	(275.13,178.70) --
	(278.41,179.96) --
	(281.68,181.31) --
	(284.95,182.74) --
	(288.22,184.26) --
	(291.49,185.88) --
	(294.77,187.58) --
	(298.04,189.39) --
	(301.31,191.29) --
	(304.58,193.29) --
	(307.85,195.40) --
	(311.12,197.62) --
	(314.40,199.94) --
	(317.67,202.37) --
	(320.94,204.91) --
	(324.21,207.57) --
	(327.48,210.35) --
	(330.76,213.25) --
	(334.03,216.26) --
	(337.30,219.41) --
	(340.57,222.67) --
	(343.84,226.07) --
	(347.11,229.60) --
	(350.39,233.26) --
	(353.66,237.06) --
	(356.93,241.00) --
	(360.20,245.07) --
	(363.47,249.29) --
	(366.74,253.65) --
	(370.02,258.16);
\definecolor{drawColor}{RGB}{0,0,0}

\path[draw=drawColor,line width= 0.4pt,line join=round,line cap=round] ( 59.48,113.63) rectangle (128.53, 65.63);
\definecolor{drawColor}{RGB}{0,255,0}

\path[draw=drawColor,line width= 0.4pt,dash pattern=on 4pt off 4pt ,line join=round,line cap=round] ( 62.18,101.63) -- ( 80.18,101.63);
\definecolor{drawColor}{RGB}{255,0,0}

\path[draw=drawColor,line width= 0.4pt,line join=round,line cap=round] ( 62.18, 89.63) -- ( 80.18, 89.63);
\definecolor{drawColor}{RGB}{0,0,255}

\path[draw=drawColor,line width= 0.4pt,line join=round,line cap=round] ( 62.18, 77.63) -- ( 80.18, 77.63);
\definecolor{drawColor}{RGB}{0,255,0}
\definecolor{fillColor}{RGB}{0,255,0}

\path[draw=drawColor,line width= 0.4pt,line join=round,line cap=round,fill=fillColor] ( 71.18,101.63) circle (  1.50);
\definecolor{drawColor}{RGB}{255,0,0}
\definecolor{fillColor}{RGB}{255,0,0}

\path[draw=drawColor,line width= 0.4pt,line join=round,line cap=round,fill=fillColor] ( 71.18, 89.63) circle (  1.50);
\definecolor{drawColor}{RGB}{0,0,255}
\definecolor{fillColor}{RGB}{0,0,255}

\path[draw=drawColor,line width= 0.4pt,line join=round,line cap=round,fill=fillColor] ( 71.18, 77.63) circle (  1.50);
\definecolor{drawColor}{RGB}{0,0,0}

\node[text=drawColor,anchor=base west,inner sep=0pt, outer sep=0pt, scale=  1.00] at ( 89.18, 98.18) {data};

\node[text=drawColor,anchor=base west,inner sep=0pt, outer sep=0pt, scale=  1.00] at ( 89.18, 86.18) {OLS fit};

\node[text=drawColor,anchor=base west,inner sep=0pt, outer sep=0pt, scale=  1.00] at ( 89.18, 74.18) {WLS fit};
\end{scope}
\end{tikzpicture}

	\caption{fit of cubic function throught the data in ex3.txt using ordinary least squares}
	\label{fig:fit-3}
\end{figure}

\subsection*{(b)}
The coeffitients fitted with the weighted least square algorithm are $[0.5003, 0.9669, -1.9655, 0.99100]$.

\subsection*{(c)}
The coefficients that are calculated using the least squares method are closer to the real values then the ones calculated using the ordinary least squares. To really say something about the difference and how well they match the given real values of the function we need more information about the distribution of the parameters.

\subsection*{(d)}
I guess I will have to adjust the derivation for ordinary least squares from the book???

\section*{Acknowledgements}
Rune Buckinx and Michaël Maex for helping to fix stupid mistakes in stupid R code
Seppe for stupid discussions about how well the questions are asked
Thibeau for organising a group crying session
Robbe for genuinly good input about trying to solve stupid questions


%\printinunitsof{in}\prntlen{\textwidth}


\end{document}
